%!TEX root = main.tex

% \section{Discussion}
% \label{sec:Discussion and limitations}
%
% In this work, we address the problem of modeling growth of real-world
% bibliographic networks. Our proposed model is an improvement over existing
% random walk growth models that preserves multiple key network structural
% properties such as degree distribution, clustering coefficient distribution and their
% joint relationship. A standard modeling assumption is that \textit{new} nodes joining a
% network can potentially make connection to \textit{any} existing node in the network in
% some prescribed manner. Our experiments suggest that local link formation
% process in which \textit{new} nodes explore local network neighborhoods
% and makes connections in the explored locality can explain multiple structural
% properties of real-world networks.
%
% We note that clustering coefficient is an important characteristic of
% real-world networks. We observe that clustering is not
% uniformly distributed over the network and clustering at nodal level to be
% highly right skewed. The skewness implies that some parts of the network is more
% clustered than the other parts. In addition to skewness, clustering at nodal
% level is correlated to nodal degree. We propose a random-walk model the that
% gives rise to prominent characteristics of the network such as skewed local
% clustering.
%
% Finally, we show, via attributed network modeling, the extensibility
% of our proposed random walk model. The modeling suggests that our
% growth model can be adapted to account for growth of other kinds of
% information such as attributes and link types. Modeling other informative
% networks such as multiplex networks and heterogeneous information networks
% should be the focus of future work.

\section{Limitations}
Now, we discuss the limitations of our work. First, our work is limited to
bibliographic datasets because of availibility of temporal data. We use the
temporal out-degree sequence of incoming nodes in the network to model the
network growth. In absence of temporal information, our growth model can be
adapted by relying on the densification power law exponent. Second, our random
walk model is sensitive to the initial graph. Since random walks explore the
locality of a network and cannot access the entire network , the initial graph
should have a giant weakly connected component. We recognise that the
intialization problem can be addressed by having non-local source of information
such as multiple seed nodes. Third, we note that our model fails to preserve
certain network properties such as path length distribution. This is because
our model does not account for nodes that serve as ``local bridges'' in the network.
Modeling local and global processes simulatenously in a joint random walk model
should lead to preservation of the discussed key network properties.
