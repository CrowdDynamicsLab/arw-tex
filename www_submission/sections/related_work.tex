%!TEX root = draft.tex
\section{Related Work}
\label{sec:Related Work}
Network growth models seek to explain a subset of structural properties observed in real
networks
We point the reader to
extensive surveys \cite{newman2003structure,albert2002statistical} of network growth models
for more information.
Note that, unlike growth models, statistical models of network replication \cite{leskovec2010kronecker,pfeiffer2014attributed}
do not model how networks grow over time and are not relevant to our work.
Below, we discuss relevant and recent work on modeling network growth.

% PA
\textbf{Preferential Attachment \& Fitness}:
In preferential attachment and fitness-based models \cite{bell2017network,medo2011temporal,bianconi2001bose,caldarelli2002scale}, a new node $u$ links to an existing node $v$
with probability proportional to the attachment function $f(k_v)$, a function of
either degree $k_v$ or fitness $\phi_v$ of node $v$.
% Node fitness is defined as a dimensionless
% measure of node attractiveness.
For instance, linear preferential attachment functions
\cite{barabasi1999emergence,kumar2000stochastic,dorogovtsev2000structure} lead to
power law degree distributions and small diameter \cite{bollobas2004diameter}
and attachment functions of degree \& node age \cite{wang2013quantifying}
can preserve realistic temporal dynamics.
Extensions of preferential
attachment \cite{mossa2002truncation,zeng2005construction,wang2009local} that
incorporate resource constraints
disregard network properties other than power law degree distribution and small diameter.
Additional mechanisms are necessary to explain network properties
such as clustering and attribute mixing patterns.

% The structural
% properties these models preserve depends on the exact definition of
% the attachment function $f$.
% The attachment function $f$ that controls edge formation in preferential
% attachment models directly impacts degree distribution
% whereas
% sublinear attachment functions \cite{krapivsky2000connectivity,dereich2013random} can generate
% networks with stretched exponential degree distribution.
% In~\Cref{sec:Experiments}, we observe that \texttt{DMS}, a preferential attachment model with
% an affine attachment function, can accurately fit the degree
% distribution of real-world networks.
% Both mechanisms disregard triadic closure and
% resource constraints.

% fitness
% In fitness growth models
% \cite{bell2017network,medo2011temporal,bianconi2001bose,caldarelli2002scale},
% the rate at which an existing node $u$ acquires links is proportional to its
% fitness $\phi_u$, a dimensionless measure of node attractiveness based on
% intrinsic nodal properties that influence edge formation. The structural
% properties a fitness-based model preserves depends on the exact definition of
% fitness.
% For example, models that define fitness as a function of degree and
% node recency \cite{wang2013quantifying} or attribute similarity
% \cite{de2013scale} can preserve realistic temporal dynamics such as popularity
% decay and assortative attribute mixing patterns.
% Since new nodes form
% each edge \textit{independently}, fitness-based models such as \texttt{HPA} do
% not incorporate triadic closure and resource constraints.
% %  Additional mechanisms %
% are necessary to preserve the high local clustering or the degree-clustering %
% correlation observed in real networks.

\textbf{Triangle Closing}:
A set of models
\cite{holme2002growing,klemm2002highly,leskovec2008microscopic}
incorporate triadic closure using triangle closing mechanisms,
which increase {average} local clustering by forming edges between nodes
with one or more common neighbors. However, as explained in~\Cref{ss:tc}, models
based on preferential attachment and triangle closing do not preserve the local
clustering of low degree nodes.

\textbf{Attributed network models}:
Attribute network growth models \cite{de2013scale,karimi2017visibility,gong2012evolution,zheleva2009co}
account for the effect of attribute homophily on edge formation and preserve mixing patterns.
Existing models can be broadly categorized as (a) fitness-based model that define fitness as a function of
attribute similarity and (b) microscopic models of network evolution that require
complete temporal information about edge arrivals \& deletion. Our experiment
results in~\Cref{sub:Experimental Results} show that well-known attributed network models
\texttt{SAN} and \texttt{KA} preserve assortative
mixing patterns, degree distribution to some extent, but not local clustering
and degree-clustering correlation.

\textbf{Random walk models}:
% Random walk models, first introduced by Vazquez \cite{vazquez2000knowing,
% vazquez2003growing}, jointly explain the emergence of multiple properties
% observed in real networks under constraints of partial access and limited
% information.
First introduced by Vazquez \cite{vazquez2000knowing}, random walk models are inherently local.
Models \cite{blum2006random} in which
new nodes only link to terminal nodes of short random walks generate
networks with power law degree distributions \cite{chebolu2008pagerank} and
small diameter \cite{mehrabian2016sa} but do not preserve clustering. Models
such as \texttt{SK} \cite{saramaki2004scale}
and \texttt{HZ} \cite{herrera2011generating}, in which new nodes probabilistically link to
each visited nodes incorporate triadic closure but are not flexible enough to preserve
{skewed} local clustering of real-world networks, as shown in~\Cref{sub:Experimental Results}.
We also observe that recursive random walk models such as \texttt{FF} \cite{leskovec2005graphs}
preserve temporal properties such as shrinking diameter but considerably overestimate local clustering
and degree-clustering relationship of real-world networks.
Furthermore, existing random walk models disregard the effect of homophily and do not model attribute mixing
patterns.

\textbf{Recent Work}:
P{\'a}lovics et al. \cite{palovics2017raising} use preferential and uniform
attachment to model the decreasing power law exponent of real-world, undirected
networks in which average degree increases over time. Singh et al.
\cite{singh2017relay} (\texttt{RL}) augment preferential attachment to explain
the shift in popularity of nodes over time via the concept of relay linking.
Both models do not incorporate mechanisms to preserve clustering, attribute
mixing patterns, and resource constraints that affect how individuals form edges
in real-world networks.

To summarize, existing models do not explain how resource constrained and local processes
\textit{jointly} preserve multiple global network properties of attributed networks.

% Growth models that account for the effect of attribute homophily on edge formation
% jointly preserve assortative mixing patterns and a few structural properties observed
% in real-world attributed networks. These models
% can be broadly categorized as (a) fitness-based model that define fitness as a
% function of attribute similarity and (b) microscopic models of network
% evolution that require complete temporal information about edge arrivals \&
% deletion. In~\Cref{sub:Experimental Results}, we observe that fitness-based
% models \cite{de2013scale,karimi2017visibility} such as \texttt{HPA} and
% microscopic models \cite{gong2012evolution,zheleva2009co} such as \texttt{SAN}
% can preserve assortative mixing patterns and degree distribution to some extent,
% but disregard clustering, degree-clustering correlation and resource
% constraints.


% Preferential attachment and fitness-based mechanisms make two strong assumptions.
% First, both mechanisms require nodes to link
% uniformly at random to \textit{any} node in the network \textit{or} have explicit
% knowledge of the degree/fitness of every node in the network.
% This assumption is
% unrealistic because individuals in real networks form edges under partial
% information and limited access constraints.
% Second, by assuming that successive edge formations are independent,
% both mechanisms
% disregard the effect of triadic closure on edge formation.
% Extensions of preferential
% attachment \cite{mossa2002truncation,zeng2005construction,wang2009local} that
% incorporate resource constraints disregard network properties other than
% power law degree distribution and small diameter.
%  by restricting access to newly added nodes or a
% small set of nodes uniformly sampled from the network
% . However, these models are
% simple, proof-of-concept methods that disregard network properties other than
% power law degree distribution and small diameter.
% Another set of models
% \cite{holme2002growing,klemm2002highly,leskovec2008microscopic}, including \texttt{HK},
% incorporate triadic closure using triangle closing mechanisms, which increase \textit{average} local
% clustering by forming edges between nodes with one or more common neighbors.
% However, as shown in~\Cref{sec:Experiments}, \texttt{HK} cannot preserve
% the \textit{skewed} clustering distribution and degree-clustering correlation
% observed in real networks.
% and contradict a key empirical finding: the
% probability of edge formation increases as a function of neighborhood overlap
% \cite{kossinets2006empirical}.


% Random walk models, first introduced by Vazquez \cite{vazquez2000knowing,
% vazquez2003growing}, jointly explain the emergence of multiple properties
% observed in real networks under constraints of partial access and limited
% information. Unlike previous mechanisms, random walk models are inherently local;
% New nodes initiate random walk(s) to explore neighborhoods of
% existing nodes and link to visited nodes. Models \cite{blum2006random} in which
% new nodes nodes only link to terminal nodes of short random walks generate
% networks with power law degree distributions \cite{chebolu2008pagerank} and
% small diameter \cite{mehrabian2016sa} but without high clustering. Models
% \cite{herrera2011generating,saramaki2004scale} such as \texttt{SK} \&
% \texttt{HZ}, in which new nodes initiate a single random walk and
% probabilistically link to visited nodes, implicitly incorporate triadic closure
% and can generate networks with high average local clustering. Moreover,
% recursive random walk models \cite{leskovec2005graphs,vazquez2003growing} such
% as \texttt{FF}, wherein nodes perform a probabilistic breadth first search to
% link to existing nodes, preserve temporal properties such as shrinking
% diameter. However, our experimental results in~\Cref{sec:Experiments} suggest
% that these models --- \texttt{HZ}, \texttt{SK} and \texttt{FF} --- do not accurately preserve
% the clustering distribution and/or degree-clustering correlation observed
% in real networks. Existing random walk models also disregard the effect of homophily on network formation.



% There has been extensive work on network growth models that \textit{explain} how
% a subset of structural properties of real world networks emerge from edge
% formation mechanisms over time. In this section, we describe four edge formation mechanisms underlying network
% growth models: preferential attachment \cite{jeong2003measuring} and its
% extensions, fitness \cite{albert2002statistical}, triangle closing mechanisms
% \cite{bianconi2014triadic} and random walks \cite{vazquez2000knowing}.  These
% edge formation mechanisms explain the emergence of multiple structural
% properties of real networks, but make one or more strong assumptions.

% Preferential attachment models such as the Barabasi-Albert model \cite{barabasi1999emergence} and Vertex
% Copying model \cite{kumar2000stochastic} explain the emergence of power law degree
% distributions of the form $p(k) = c\cdot k^{-\alpha}$, commonly observed in real
% networks. In preferential attachment models, new nodes that join the network
% form edges to existing nodes with probability proportional to their degree. This
% implies that high degree nodes accumulate edges quicker than low degree nodes.
% An intuitive explanation of preferential attachment is that new nodes are more
% likely to link to ``popular'' high degree nodes than relatively unknown, low
% degree nodes. However, preferential attachment implicitly assumes that edge
% formation depends only on degree and cannot explain why real networks exhibit
% high clustering or degree distributions that do follow power law.

% Unlike preferential attachment models, fitness models are flexible enough to
% generate networks with varying degree distributions and degree correlation. The
% inability of preferential attachment to preserve multiple structural properties
% suggests that factors other than degree influence edge formation. In fitness
% models, new nodes that join the network form edges to existing nodes with
% probability proportional to their fitness. The fitness $\phi_i $ of node $v_i$
% is a function of intrinsic nodal properties that influence edge formation. The
% structural properties a fitness model preserves depends on the exact definition
% of fitness. For example, the fitness model introduced in \cite{bianconi2001bose} increases fitness
% as a function of degree and node recency. This can preserve temporal dynamics
% such as decay in popularity \cite{wang2013quantifying} of old nodes in citation networks. Simpler fitness
% models can generate degree distributions that follow power law, exponential or
% lognormal \cite{medo2011temporal} distributions. However, since new nodes form each edge
% \textit{independently}, fitness alone cannot explain the emergence of high local
% clustering or the bivariate relationship between degree and clustering observed
% in real networks.

% Edge formation mechanisms proposed by the above network growth models make
% two strong assumptions.
% \begin{itemize}
%     \item \textbf{Complete access to information}
%     These mechanisms require nodes to link uniformly at random to \textit{any}
%     node in the network or have explicit knowledge of the degree/fitness of
%     every node in the network. This assumption is unrealistic because nodes in
%     real networks form edges partial information and limited access constraints.
%
%     \item \textbf{Successive edge formations are independent} There is a strong, implicit
%     assumption that a node's decision to link to another node is independent of the
%     nodes to which it has already linked. This assumption contradicts a key empirical
%     finding that the probability of edge formation increases as a function
%     of neighborhood overlap in social, information and citation networks.
% \end{itemize}
%
% % triangle closing
% Extensions of preferential attachment and fitness models \cite{holme2002growing,klemm2002highly} using triangle
% closing mechanisms explain why social \& information networks have high
% average local clustering \cite{newman2001clustering}. In these models, a new node that joins the network
% ``closes triangles'' by linking to neighbors of nodes it has already linked to
% based on degree or fitness. Closing triangles increases the number of edges
% between neighbors, thereby increasing the average local clustering. Triangle
% closing mechanisms essentially model triadic closure, a sociological process
% that explains why two nodes with mutual neighbor(s) have an increased
% probability of connection. In~\Cref{sec:Experiments}, we show that these
% models are not flexible enough to capture the skewness and variance in
% clustering distributions of real networks.
%
% % limited info
% A few models \cite{mossa2002truncation,zeng2005construction,wang2009local} adapt
% preferential attachment and fitness to model network growth under constraints of
% limited access and information. These models incorporate constraints by
% restricting access to recent nodes or a small set of nodes uniformly sampled
% from the network. However, these simple models are proof-of-concept methods that
% do not generate networks with varying degree distributions and realistic local
% clustering distributions.

% Random walk models jointly explain multiple structural properties of real
% networks under constraints of limited access and information. New nodes explore
% neighborhoods of existing nodes without any assumption of global information and
% use simple rules to form edges. New nodes that join the network perform one or
% more random walks to link to existing nodes. For example, the Random Surfer
% model \cite{blum2006random}, in which new nodes link to the terminal nodes of short random walks,
% generate networks that exhibit power law degree distributions. Importantly, this
% model explains preferential attachment as an \textit{emergent} property of local
% proceses. Random walk models in which new nodes perform random walk(s) and link
% to \text{any} visited node incorporate triadic closure and generate networks
% with heavy tailed degree distribution and high local clustering \cite{herrera2011generating}. In~\Cref{sec:Experiments}, we show that models based on random walks outpeform
% well-known \textit{global} edge formation mechanisms in preserving structural
% properties of citation networks. However, current random walk models are
% either inflexible or too simple to accurately capture local clustering observed
% in real networks.

% % summary
% To summarize, network growth models use global or local edge formation mechanism(s) to
% explain structural properties of real networks. Structural properties preserved
% by global edge formation mechanisms such as preferential attachment can be
% preserved by local processes such as random walks as well. However, unlike
% random walks, extensions of global processes such as preferential attachment \&
% fitness models make strong, unrealistic assumptions.