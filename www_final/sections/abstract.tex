%!TEX root = ../draft.tex

This paper proposes an attributed network growth model. Despite the knowledge
that individuals use limited resources to form connections to similar others, we
lack an understanding of how local and resource-constrained mechanisms explain
the emergence of rich structural properties found in real-world networks. We
make three contributions. First, we propose an interpretable and accurate model
of attributed network growth that jointly explains the emergence of in-degree
distribution, local clustering, clustering-degree relationship and attribute
mixing patterns. Second, we make use of biased random walks to develop a model
that forms edges locally, without recourse to global information. Third, we
account for multiple sociological phenomena---bounded rationality; structural
constraints; triadic closure; attribute homophily; preferential attachment.
We explore the parameter space of the proposed Attributed Network Growth (\texttt{ARW})
to show each model parameter intuitively modulates network structure.
Our experiments show that \texttt{ARW} accurately preserves network structure and attribute mixing patterns of
six real-world networks; it improves upon the performance of eight
well-known models by a significant margin of
2.5--$10\times$.

% However, well-known growth models that preserve multiple structural properties do not
% incorporate these resource constraints. Conversely, resource constrained growth models
% cannot jointly preserve multiple structural
% properties of real networks. Furthermore, most growth models disregard
% the effect of homophily on edge formation and global network structure.
% Our Attributed Random Walk (\texttt{ARW}) model explains how structural \&
% content-based properties of real-world networks jointly arise from individual
% preferences \& edge formation under constraints of limited information and network access.

% In our model, each node that joins the network selects a seed node from which it initiates a
% biased random walk to concurrently explore the network and link to existing nodes.
% At each step of the walk, the new node either jumps back to
% the seed node or chooses an outgoing or incoming edge to visit another node; It
% probabilistically links to each visited node and halts after forming a few edges
% Through our experiments, we observe that the proposed model \texttt{ARW} preserves
% network structure and attribute mixing patterns of six real-world networks