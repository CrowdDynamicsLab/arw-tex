%!TEX root = ../main.tex

\section{Related Work}
\label{sec:Related Work}

Preferential attachment and fitness-based models \cite{bell2017network,medo2011temporal,bianconi2001bose,caldarelli2002scale}
can preserve heavy-tailed degree distribution, small diameter \cite{bollobas2004diameter} and temporal dynamics \cite{wang2013quantifying}
of real-world networks. Extensions of preferential attachment \cite{mossa2002truncation,zeng2005construction,wang2009local} that account for
partial network access disregard properties such as clustering and mixing patterns.
Models
\cite{holme2002growing,klemm2002highly,leskovec2008microscopic}
couple preferential attachment with triangle closing to incorporate triadic closure.
This increases {average} local clustering by forming edges between nodes
with one or more common neighbors, as shown in~\Cref{sec:Experiments}.
% but does not
% accurately preserve distributional properties of local clustering.


Models \cite{de2013scale,karimi2017visibility,gong2012evolution,zheleva2009co}
that account for attribute mixing can be largely categorized as (a) fitness-based model that define fitness as a function of
attribute similarity and (b) "microscopic" growth models  that require
complete temporal information about edge insertions \& deletion.
In~\Cref{sub:Experimental Results}, we show that attributed network models
\texttt{SAN} and \texttt{KA} preserve mixing patterns but do not account for other
structural properties of real-world networks.

First introduced by Vazquez \cite{vazquez2000knowing}, random walk models are inherently local.
Models \cite{blum2006random} in which
new nodes only link to terminal nodes of short random walks generate
networks with power law degree distributions \cite{chebolu2008pagerank} and
small diameter \cite{mehrabian2016sa} but do not preserve clustering. Models
such as \texttt{SK} \cite{saramaki2004scale}
and \texttt{HZ} \cite{herrera2011generating}, in which new nodes probabilistically link to
each visited nodes incorporate triadic closure but are not flexible enough to preserve
{skewed} local clustering, as shown in~\Cref{sub:Experimental Results}.
Recursive random walk models such as \texttt{FF} \cite{leskovec2005graphs}
preserve temporal properties such as shrinking diameter but considerably overestimate local clustering.
Furthermore, existing random walk models do not account for nodal attributes.

To summarize, existing models do not accurately explain how resource constrained processes
shape well-defined global properties of attributed networks over time.
Please refer to the
extended version of the paper \cite{shah2017growing} for a detailed review of existing work.
