%!TEX root = ../main.tex
\section{Conclusion}
\label{sec:Conclusion}
In this paper, we proposed a simple, interpretable model of attributed network
growth. \texttt{ARW} grows a directed network in the following manner: an
incoming node selects a seed node based on attribute similarity, initiates a
biased random walk to explore the network by navigating through neighborhoods of
existing nodes, and halts the random walk after connecting to a few visited
nodes. To the best of our knowledge, \texttt{ARW} is the first model that
unifies multiple sociological phenomena---bounded rationality; structural
constraints; triadic closure; attribute homophily; preferential
attachment---into a single local process to model global network structure
\textit{and} attribute mixing patterns.
We explored the parameter space of the model to show how each parameter
intuitively controls one or more key structural properties.
Our experiments on six
large-scale citation networks showed that \texttt{ARW} outperforms
relevant and recent existing models by a statistically significant
factor of 2.5--$10\times$.

% We plan to extend the \texttt{ARW} model in three ways: modeling undirected, social
% networks, understanding the
% emergence of higher-order clustering \cite{yin2018higher} and modeling the
% effect of homophily on the formation of temporal motifs \cite{paranjape2017motifs}

% In this paper, we propose a network growth model that explains the
% structure of attributed networks through a local edge formation mechanism. Our
% model \texttt{ARW} is normative, accurate and simple. We incorporate multiple
% sociological phenomena into our model to intuitively prototype how individuals
% form edges under constraints of limited information and partial network access.
% Through our experiments, we validate the efficacy of our model in jointly preserving
% multiple structural properties and attribute mixing patterns of real-world networks.
% Our work signifies the need to understand how local processes of link formation
% give rise to structural characteristics of real-world networks.

% We also show that our
% model can preserve local assortativity distributions of attributed networks.
% Furthermore, we discussed the weaknesses of global processes such as
% preferential attachment \& triangle closing and addressed the limitations of our
% model.

% We identify three future directions:

% In this paper, we model resource-constrained network growth model in which nodes
% use a random walk process to form edges under constraints of limited information
% and network access constraints. The problem is important because edge formation
% in real networks is usually a local process. In typical network growth
% scenarios, nodes in the network either have limited information about the other
% nodes in the network or the system allows access to only restricted portion of
% the existing network. It therefore becomes imperative to model how the local
% processes of link formation gives rise to network characteristics. In this work,
% we show that multiple structural properties of real networks can arise from the
% local process of exploration and link formation. Our results indicate significant
% improvement over the next best competing model \textsc{HZ}
% \cite{herrera2011generating} by a significant margin.
