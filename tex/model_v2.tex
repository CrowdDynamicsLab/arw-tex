%!TEX root = draft.tex

\section{Attributed Random Walk Model}
\label{sec:Proposed Model}
We propose the Attributed Random Walk (\texttt{ARW}) model to explain the emergence
of key structural properties of real-world networks through \textit{entirely local}
edge formation mechanisms.
\texttt{ARW} grows a directed network over time as new nodes join the network.
The mechanism that incoming nodes use to form edges intuitively corresponds to how we
expect researchers to conduct a literature survey and cite relevant work.
First, the researcher broadly identifies one or more \textit{relevant} papers,
possibly with the help of external information sources.
Then, under time and information constraints, the individual navigates a chain
of references to identify \textit{similar} papers that either support or
address the problem in hand.  Next, through careful analysis, she
decides to cite a subset of these papers.
Similarly, an incoming node selects a seed node and initiates a random walk to
explore the network by navigating through neighborhoods of existing nodes. It halts
the random walk after connecting to a few visited nodes.

In this section, we first describe the edge formation mechanisms underlying \texttt{ARW}.
Then, we explain how \texttt{ARW} intuitively incorporates multiple sociological phenomena.
Finally, we briefly discuss the methods required to fit the model to data.

\subsection{Model Details}
\label{sub:Model Description}


The Attributed Random Walk (\texttt{ARW}) model grows a directed network $\{\hat{G}_t\}^T_{t=1}$
in $T$ time steps.
More formally, at every discrete time step $t$, a
new node $u$, with attribute value $B(u)$, joins the network $\hat{G}_t$.
After joining the network, node $u$ forms $m(t)$ edges to
existing nodes.
% At time $t$, $G_t$ consists of ${|V_t|=|V_0|+t}$ nodes,
% ${|E_t|=|E_{t-1}|+m(t)}$ edges and the set of attribute values ${A_t = A_{t-1}
% \cup \{A(u)\}}$.
The outdegree of incoming nodes increases over time to
reflect the nonlinear growth and densification of real networks.
% We discuss the
% issue of initializing $G_0$, sampling attribute values of inomcing nodes and modeling
% densification in \Cref{sub:Model Fitting}.

The edge formation mechanism consists of two components: \textsc{Select-Seed} and
\textsc{Random-Walk}. A new node $u$ with attribute value $B(u)$ that joins the
network at time $t$ first selects a seed node $S(u)$ using \textsc{Select-Seed}:
\\\\
\tikzstyle{background rectangle}=[thin,draw=black]
\begin{tikzpicture}[show background rectangle]
\node[align=left, text width=.93\linewidth, inner sep=.5em]{
(1) With probability $p_a$, randomly select $S(u)$ from the set of existing nodes that have attribute
value $B(u)$.

\vspace{1mm}
(2) Otherwise, with probability $1-p_a$, randomly select $S(u)$ from the set of existing nodes that
do \textit{not} have attribute value $B(u)$.
};
\node[xshift=3ex, yshift=-.7ex, overlay, fill=white, draw=white, above
right] at (current bounding box.north west) {
\textsc{Select-Seed}
};
\end{tikzpicture}

% The attribute parameter $p_a$ incorporates the attribute preferences of incoming nodes
% into the model.

\textsc{Seed-Select} accounts for homophilic preferences of incoming nodes using
attribute parameter $p_a$. As shown in \cref{fig:randomwalk}, after selecting
the seed $S(u)$, node $u$ initiates a
random walk using \textsc{Random-Walk} to form $m(t)$ links.
The \textsc{Random-Walk} mechanism consists of four parameters - $\alpha$ \& $p_a$
parameterize edge formation decisions and $p_j$ \& $p_o$ characterize random walk
traversals:
\\\\
\tikzstyle{background rectangle}=[thin,draw=black]
\begin{tikzpicture}[show background rectangle]
\node[align=left, text width=.93\linewidth, inner sep=.5em]{
(1) At each step of the walk, new node $u$ visits node $v_i$.
    \begin{itemize}
        \item If $B(u)=B(v_i)$, $u$ links to $v_i$ with probability $\alpha \cdot p_a$
        \item Otherwise, $u$ links to $v_i$ with probability $\alpha \cdot (1-p_a)$
    \end{itemize}

\vspace{1mm}
(2) Then, with probability $p_j$, $u$ jumps back to seed $s_u$.

\vspace{1mm}
(3) Otherwise, with probability ${1-p_j}$, $u$ continues to walk. It picks an outgoing edge with probability $p_o$ \textit{or}
an incoming edge with probability $1-p_o$ to visit a neighbor of $v_i$.

\vspace{1mm}
(4) Steps 1-3 are repeated until $u$ links to $m(t)$ nodes.
};
\node[xshift=3ex, yshift=-.7ex, overlay, fill=white, draw=white, above
right] at (current bounding box.north west) {
\textsc{Random-Walk}
};
\end{tikzpicture}

When attribute data is absent, the attribute parameter $p_a$ is not required.
Then, \textsc{Seed-Select} simply selects an existing node uniformly at random
and the probability of edge formation in \textsc{Random-Walk} is equal to
the rate parameter $\alpha$ only.

Next, we explain how each parameter is necessary to conform to normative
behavior of individuals in evolving networks.

\subsection{\texttt{ARW} \& Normative Behavior}
\label{sub:Model Interpretation}

The Attributed Random Walk model unifies multiple well-known sociological phenomena
into its edge formation mechanisms \textsc{Seed-Select} \& \textsc{Random-Walk}.

\newtheorem{ph}{Phenomenon}

\begin{ph}
(Limited Resources) Individuals are boundedly rational \cite{simon1972theories,gigerenzer1996reasoning,lipman1995information}
actors that form edges under constraints of limited information, partial network access and finite cognitive capacity.
\end{ph}
In \texttt{ARW}, we use random walks to incorporate constraints of limited information
and partial network access. A new node $u$ selects a seed node from which it
initiates a biased random walk. Then, $u$ uses simple rules to connect to each visited
nodes probabilistically and halts the walk after forming a few edges, as shown in
\cref{fig:randomwalk}. Random walks inherently account for limited information
and partial network access as they only require information about the
1-hop neighborhood of visited nodes.

\begin{ph}
(Structural Constraints) Structural factors such as network distance
act as constraints that limit edge formation to proximate nodes.  \cite{35626}
\end{ph}

We incorporate structural constraints into \texttt{ARW} using the jump parameter $p_j$.
The jump parameter $p_j$ is the probability which a new node jumps back to its seed node
after each step of the random walk. This implies that the probability with which the new node
is at most $k$ steps from its seed node is $1-p^k_j$; As a result, the jump parameter $p_j$
controls the extent to which new nodes' random walks explore the network to form edges.

\begin{ph}
(Triadic Closure) Nodes with common neighbors have an
increased likelihood of forming a connection. \cite{simmel1950sociology}
\end{ph}

We control the effect of triadic closure on edge formation using the
rate parameter $\alpha$. A new node $u$ uses a random walk to
link to each visited node with probability proportional to $\alpha$. As a
result, the probability with which node $u$ closes a triad by linking to
a visited node and its neighbor is proportional to $\alpha^2$.

\begin{ph}
(Attribute Homophily) Nodes that have similar attributes are more likely
to form a connection. \cite{mcpherson2001birds}
\end{ph}
We incorporate attribute homophily into the edge formation process via attribute parameter $p_a$. New node
$u$ links to each visited node $v$ with probability $\alpha \cdot p_a$ if they share
the same attribute value. Otherwise, $u$ connects to $v$ with probability $\alpha \cdot (1-p_a)$.
The attribute parameter $p_a$ effectively controls the global assortativity coefficient.

\begin{ph}
(Preferential Attachment) Nodes tend to link to high degree nodes that have more
visibility. \cite{barabasi1999emergence}
\end{ph}
% Individuals cannot link to high degree nodes \textit{directly} under constraints of limited information
% and partial network access.
In the absence of global information, we induce preferential attachment
in \texttt{ARW} by adding structural bias to random walk traversals. We utilize the positive correlation
between node age and node degree to adjust bias towards visiting old nodes that tend to have high degree.
Indeed, random walks that traverse outgoing edges only eventually visit old nodes that tend to have high indegree.
Similarly, random walks that traverse incoming edges only visit recently joined nodes that tend to have low indegree.
We use out parameter $p_o$, the probability with which nodes choose outgoing edges in their
random walks, to adjust the effect of preferential attachment on edge formation.

\texttt{ARW} unifies five well-known sociological phenomena into a single edge
formation mechanism based on random walks. Random walks inherently account for
limited information and partial network access. Furthermore, the jump parameter $p_j$, attribute parameter $p_a$,
rate parameter $\alpha$ and out parameter $p_o$ incorporate the effect of structural constraints,
homophily, triadic closure and preferential attachment respectively.



\subsection{Model Fitting}
\label{sub:Model Fitting}

We now briefly describe methods to estimate model parameters,
initialize $\hat{G}$ at time ${t=0}$, densify $\hat{G}$
over time and sample incoming nodes' attribute values.

\textit{Parameter Estimation}.
% The rate parameter $\alpha$, attribute parameter $p_a$, jump parameter $p_j$ and
% out parameter $p_o$ jointly control the edge formation mechanism in \texttt{ARW}.
% These parameters subsequently determine the structural properties of the network $\hat{G}$
% generated by \texttt{ARW}.
The parameter estimation task consists of finding the set of
parameters values for $(\alpha, p_a, p_j, p_o)$ that best explain the structural properties
of an observed network $G$. We use a straightforward grid search method to estimate
the four parameters. Other derivative-free optimization methods such as the Nelder-Mead \cite{nelder1965simplex}
method can be used to quicken parameter estimation.
% We describe the evaluation metrics and selection
% criteria in \Cref{sub:Experimental Setup}.

\textit{Initialization}. The edge formation mechanism in \texttt{ARW} is
sensitive to a large number of weakly connected components (\texttt{WCC}s) in the
initial network $\hat{G}_0$ because incoming nodes can only form edges to nodes
in the same \texttt{WCC}. To ensure that $\hat{G}_0$ is weakly
connected, we perform an undirected breadth-first search on the observed,
to-be-fitted network $G$ that starts from the oldest node and terminates after
visiting $0.1\%$ of the nodes. The initial network $\hat{G}_0$ is the small \texttt{WCC}
induced from the set of visited nodes.
% Simpler initialization methods
% such as sampling $\hat{G}_0$ from the Erdos-Renyi model or Watts-Strogatz model
% yield similar results.

\textit{Node Outdegree}.
% In \cref{sec:Analysis}, we observed that real
% networks densify over time, with the number of edges growing superlinearly in
% the number of nodes.
The outdegree of incoming nodes increases over time in real-world networks.
We incorporate this phenomenon in \texttt{ARW}
to coarsely reflect the rate of growth in the observed network $G$.
Each incoming node $u$ that joins $\hat{G}$ at time $t$ corresponds to some
node that joins the observed network $G$ in year $y(t)$; The number of edges $m(t)$
that $u$ forms is equal to the average outdegree of nodes that join $G$ in year $y(t)$.

\textit{Sampling Attribute Values}. In real networks $G=(V,E,B)$,
the distribution over the set of attribute values $P_{\textsc{g}}(B)$ changes over time.
For instance, the attribute distribution over journals in the \texttt{APS} citation
network changes over time as old journals decay in popularity and new journals gain traction.
The change in the attribute distribution over time is an exogenous factor and varies for every network.
To incorporate this phenomenon into \texttt{ARW}, we sample the attribute value $B(u)$ of node $u$, that
joins $\hat{G}$ at time $t$, from $P_{\textsc{g}}(B{\mbox{ | year}=y(t)})$, the observed attribute distribution
conditioned on the corresponding year of node $u$.


To summarize, the Attributed Random Walk (\texttt{ARW}) model
intuitively describes how individuals form edges under resource constraints.
\texttt{ARW} uses four parameters --- $\alpha$, $p_a$, $p_j$, $p_o$ --- to incorporate
individuals' biases towards similar, proximate and high degree nodes.
Next, our experiments in
\cref{sec:Experiments} show that $\texttt{ARW}$ accurately preserves
\textit{multiple} structural properties of real networks
