%!TEX root = draft.tex
\section{Problem Statement}
\label{sec:Problem Statement}

Consider an attributed directed network $G=(V,E,B)$, where $V$ \& $E$ are
sets of nodes \& edges and each node has an attribute value $b \in B$.
The goal is to develop a directed network growth model that preserves structural
and attribute based properties observed in $G$. The growth model should be
normative, accurate and parsimonious:
\begin{enumerate}
    \item \textbf{Normative}: The model should account for normative behavior. In real-world
    networks, multiple sociological phenomena influence how individuals form edges under
    constraints of limited global information and under partial network access.
    \item \textbf{Accurate}: The model should preserve key structural
    and attribute based properties such as heavy tailed degree distribution, skewed
    local clustering, negatively correlated degree-clustering relationship
    and attribute mixing patterns.
    \item \textbf{Parsimonious}: The model should have as few parameters as possible, but be expressive enough to generate networks with varying structural properties.
\end{enumerate}

Next, we present extensive empirical analysis on real-world datasets to motivate our attributed random walk model.

% \harshay{Normative or Intuitive?}

% Extensive research on network growth has led to development of well-known growth
% models that generate realistic networks. However, the edge formation mechanisms
% of most network growth models tend to make strong assumptions about either
% knowledge (e.g. complete degree/fitness distribution known) or access (e.g. pick
% nodes uniformly at random).
%
% The goal of this paper is to model network growth under information and resource
% constraints using edge formsation mechanisms. The growth model should be able to
% jointly explain global structural properties of real networks such as degree
% distribution, clustering coefficient distribution, degree-clustering
% relationship and degree correlations The model should incorporate information \&
% resource constraints that influence edge formation in real networks.
