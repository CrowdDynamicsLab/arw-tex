%!TEX root = draft.tex

We propose a network growth model based on local processes that jointly explains the
emergence of key structural properties of real-world attributed directed networks:
heavy-tailed indegree distribution, attribute mixing patterns, high local
clustering and degree-clustering correlation.
In real-world networks, individuals form edges
under constraints of limited information and partial network access. However,
well-known growth models that preserve multiple structural properties do not
incorporate these resource constraints. Conversely, resource constrained growth models
cannot jointly preserve multiple structural
properties of real networks. Furthermore, most growth models disregard
the effect of homophily on edge formation and global network structure.
Our Attributed Random Walk (\texttt{ARW}) model explains how structural \&
content-based properties of real-world networks jointly arise from individual
preferences \& edge formation under constraints of limited information and network access.
In our model, each node that joins the network selects a seed node from which it initiates a
biased random walk to concurrently explore the network and link to existing nodes.
% At each step of the walk, the new node either jumps back to
% the seed node or chooses an outgoing or incoming edge to visit another node; It
% probabilistically links to each visited node and halts after forming a few
% edges.
Our experimental results against eight well-known growth models
indicate significant improvement (2.5-10x) in accurately preserving global
structural properties and attribute mixing patterns of
six large scale real-world networks.