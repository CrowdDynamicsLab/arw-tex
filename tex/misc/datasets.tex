%!TEX root = draft.tex

\section{Datasets}
\label{sec:Datasets}

We consider six large-scale citation networks from diverse sources: research articles,
utility patents and judicial cases.
We study the structural
and content properties of these networks in \Cref{sec:Analysis} and empirically
validate the effectiveness of the proposed model using these network datasets in~\Cref{sec:Experiments}.

We focus on citation networks for three reasons. First, nodes in citation networks
form all outgoing edges to existing nodes at the time of joining the network; Nodes do
not form or delete edges at a later time. This allows us to analyze
the edge formation mechanisms of new nodes that join the network form edges.
% Other edge dynamics such as edge deletion and addition of edges between existing
% nodes are important and we plan to investigate them at a later time.
Second, citation network datasets include the time (e.g., publication year of academic
papers) at which nodes join the network. As a result, local edge formation
processes and global structural properties can be better understood by studying
network snapshots at different stages of the growth process. Third, the citation
networks are large networks that tend to have one or more nodal attributes (e.g. category of patents)
and span multiple decades. As a result, the structural and content properties of the citation
networks considered are well-defined.

\begin{table}[H]
 {
  \begin{tabular}[c]{lrrrcr} \toprule
   Network & $|V|$           & $|E|$        & $T$        & $A$              & $|A|$ \\ \midrule
   \texttt{USSC}         & 30,288     & 216,738      & 1754-2002  & - & -                  \\
   \texttt{HEP-PH}       & 34,546     & 421,533      & 1992-2002  & - & -                  \\
   \texttt{Semantic}     & 7,706,506  & 59,079,055   & 1991-2016  & - & -                  \\   \midrule
   \texttt{ACL}          & 18,665     & 115,311      & 1965-2016  & \textsc{venue} & 50    \\
   \texttt{APS}          & 577,046    & 6,967,873    & 1893-2015  & \textsc{journal} & 13   \\
   \texttt{Patents}      & 3,923,922  & 16,522,438   & 1975-1999  & \textsc{category} & 6  \\
   % \texttt{PYPI}         & 25,169     & 71,371       & 2002-2018  & \textsc{category} & 9  \\
  \bottomrule
  \end{tabular}
  \vspace{1mm}
  \caption{Network summary statistics: number of nodes $|V|$ and edges $|E|$, time period
  $T$, categorical attribute $A$ and number of attribute values $|A|$ of seven citation networks.}
  \label{table:datasets}
 }
\end{table}

Now, we briefly describe the datasets considered in this paper. Three of the six
network datasets have nodal attribute data; That is, each node has a categorical attribute value.
\Cref{table:datasets} provides summary statistics of the following networks:
\begin{enumerate}
    % (V,E) = (22049, 138871), used (V,E): 18665 115358
    \item{\textbf{Association of Computational Linguistics}} (\texttt{ACL}) \cite{acldata} is an attributed academic citation network
    that consists of papers published in ACL conferences, journals and workshops.
    The attribute value of each paper is the name of the venue where it was published.

    % (V,E) = (22049, 138871), used (V,E): 18665 115358
    % \item{\textbf{Python Package Index}} (\texttt{PYPI}) \footnote{http://web.stanford.edu/class/cs224w/resources.html} is an attributed dependency graph of Python
    % software packages. Each software package is associated with a category.

    % entire dataset used
    \item{\textbf{U.S. Supreme Court Cases}} (\texttt{USSC}) \cite{fowler2008authority} is a judicial citation network of
    U.S. Supreme Court cases. There is an edge from case $i$ to case $j$ if and only if case $i$ cites case $j$ in its majority opinion.

    % used: (V,E) = (30,558, 347,228) after removing nodes w. missing time data
    \item{\textbf{ArXiv HEP-PH}} (\texttt{HEP-PH}) \cite{gehrke2003overview} is an academic citation network of HEP-PH (high energy
    physics phenomenology) papers in the ArXiv e-print.

    % used: (V,E) = (556661, 6647769) after removing nodes w. missing time + categorial data
    \item{\textbf{APS Journals}} (\texttt{APS}) \footnote{https://journals.aps.org/datasets} is an attributed academic citation network maintained by
    the American Physical Society (\texttt{APS}).
    The attribute value of each paper is the \texttt{APS} journal in which it was published.

    % used: (V,E) = (2047881, 10088564) after removing nodes w. missing time + categorical data
    \item{\textbf{U.S Utility Patents}} (\texttt{Patents}) \cite{leskovec2005graphs} is an attributed citation network of U.S. utility patents maintained by
    the National Bureau of Economic Research (NBER).
    The attribute value of each patent is an NBER patent category.

    % used: (V,E) = (5987642, 45028807) after removing nodes w. missing time data
    \item{\textbf{Semantic Scholar}} (\texttt{Semantic}) \cite{ammar} is an academic citation network of
    Computer Science and Neuroscience papers, released in June 2017 by Semantic Scholar.
\end{enumerate}

In this section, we outlined the citation network datasets that we use in our analysis and experiments.
Next, we discuss common factors that affect edge formation mechanisms and identify common global structural
properties of real networks.

% we briefly reviewed edge formation processes and related global network properties.
% Then, we outlined the citation network datasets that we use in our analysis and experiments. In
% the next section, we present the problem statement.