%!TEX root = draft.tex
\section{Empirical Analysis}
\label{sec:Analysis}

In this section, we begin by describing six large-scale network datasets that we use in our
analysis and experiments. Then, we
describe key factors that impact edge formation and analyze global structural
properties of real-world networks. Finally, we briefly discuss insights from
empirical studies in sociology and common assumptions in network modeling.

\subsection{Datasets}
\label{sec:Datasets}

We consider six large-scale citation networks from diverse sources: research articles,
utility patents and judicial cases.
We focus on citation networks for three reasons. First, nodes in citation networks
form all outgoing edges to existing nodes at the time of joining the network; Nodes do
not form or delete edges at a later time. This allows us to analyze
the edge formation mechanisms of new nodes that join the network form edges.
% Other edge dynamics such as edge deletion and addition of edges between existing
% nodes are important and we plan to investigate them at a later time.
Second, citation network datasets include the time (e.g., publication year of academic
papers) at which nodes join the network. As a result, local edge formation
processes and global structural properties can be better understood by studying
network snapshots at different stages of the growth process. Third, the citation
networks are large networks that tend to have one or more nodal attributes (e.g. category of patents)
and span multiple decades. As a result, the structural and content properties of the citation
networks considered are well-defined.

\begin{table}[b]
 {
  \begin{tabular}[c]{lrrrcr} \toprule
   Network & $|V|$           & $|E|$        & $T$        & $A$              & $|A|$ \\ \midrule
   \texttt{USSC}         & 30,288     & 216,738      & 1754-2002  & - & -                  \\
   \texttt{HEP-PH}       & 34,546     & 421,533      & 1992-2002  & - & -                  \\
   \texttt{Semantic}     & 7,706,506  & 59,079,055   & 1991-2016  & - & -                  \\   \midrule
   \texttt{ACL}          & 18,665     & 115,311      & 1965-2016  & \textsc{venue} & 50    \\
   \texttt{APS}          & 577,046    & 6,967,873    & 1893-2015  & \textsc{journal} & 13   \\
   \texttt{Patents}      & 3,923,922  & 16,522,438   & 1975-1999  & \textsc{category} & 6  \\
   % \texttt{PYPI}         & 25,169     & 71,371       & 2002-2018  & \textsc{category} & 9  \\
  \bottomrule
  \end{tabular}
  \vspace{1mm}
  \caption{Summary statistics of six network datasets: number of nodes $|V|$ and edges $|E|$, time period
  $T$, categorical attribute $A$ and number of attribute values $|A|$.}
  \label{table:datasets}
 }
\end{table}

Now, we briefly describe the datasets considered in this paper. Three of the six
network datasets have nodal attribute data; That is, each node has a categorical attribute value.
\Cref{table:datasets} provides summary statistics of the following networks:
\begin{enumerate}
    % (V,E) = (22049, 138871), used (V,E): 18665 115358
    \item{\textbf{Association of Computational Linguistics}} (\texttt{ACL}) \cite{acldata} is an attributed academic citation network
    that consists of papers published in ACL conferences, journals and workshops.
    The attribute value of each paper is the name of the venue where it was published.

    % (V,E) = (22049, 138871), used (V,E): 18665 115358
    % \item{\textbf{Python Package Index}} (\texttt{PYPI}) \footnote{http://web.stanford.edu/class/cs224w/resources.html} is an attributed dependency graph of Python
    % software packages. Each software package is associated with a category.

    % entire dataset used
    \item{\textbf{U.S. Supreme Court Cases}} (\texttt{USSC}) \cite{fowler2008authority} is a judicial citation network of
    U.S. Supreme Court cases. There is an edge from case $i$ to case $j$ if and only if case $i$ cites case $j$ in its majority opinion.

    % used: (V,E) = (30,558, 347,228) after removing nodes w. missing time data
    \item{\textbf{ArXiv HEP-PH}} (\texttt{HEP-PH}) \cite{gehrke2003overview} is an academic citation network of HEP-PH (high energy
    physics phenomenology) papers in the ArXiv e-print.

    % used: (V,E) = (556661, 6647769) after removing nodes w. missing time + categorial data
    \item{\textbf{APS Journals}} (\texttt{APS}) \footnote{https://journals.aps.org/datasets} is an attributed academic citation network maintained by
    the American Physical Society (\texttt{APS}).
    The attribute value of each paper is the \texttt{APS} journal in which it was published.

    % used: (V,E) = (2047881, 10088564) after removing nodes w. missing time + categorical data
    \item{\textbf{U.S Utility Patents}} (\texttt{Patents}) \cite{leskovec2005graphs} is an attributed citation network of U.S. utility patents maintained by
    the National Bureau of Economic Research (NBER).
    The attribute value of each patent is an NBER patent category.

    % used: (V,E) = (5987642, 45028807) after removing nodes w. missing time data
    \item{\textbf{Semantic Scholar}} (\texttt{Semantic}) \cite{ammar} is an academic citation network of
    Computer Science and Neuroscience papers, released in June 2017 by Semantic Scholar.
\end{enumerate}

Next, we study the structural and content properties of these networks in \cref{subsec:factors} and empirically
validate the effectiveness of the proposed model using these network datasets in~\Cref{sec:Experiments}.
%
% In this section, we outlined the citation network datasets that we use in our analysis and experiments.
% Next, we discuss common factors that affect edge formation mechanisms and identify common global structural
% properties of real networks.


\subsection{Observations from Network Data}
\label{subsec:factors}

% Factors that influence edge formation at the nodal level have a cumulative
% effect on global structural properties of real-world networks.
Compact
statistical descriptors of global network properties ~\cite{newman2010networks}
such as degree distribution, local clustering and attribute assortativity
quantify the extent to which local edge formation phenomena shape global network
structure.

\textbf{Preferential Attachment \& Degree Distribution}
In the preferential attachment process  \cite{simon1955class,barabasi1999emergence},
nodes with higher degree receive links at a faster rate because incoming nodes tend to
link to well-connected nodes that have more visibility. As a result, initial differences in node
connectivity get reinforced over time, giving rise to a rich-get-richer effect.
This phenomenon  explains why the citation networks datasets exhibit heavy
tailed degree distributions; It also implies that most papers receive zero or a few
citations, but a small but significant percent of the nodes turn into popular
hubs that receive many citations. Log-normal fits describe the indegree
distribution of all network datasets, well consistent with Broido \& Clauset's
\cite{broido2018scale} observation that real-world networks with truly power law
degree distributions are rare; The parameters of the log-normal fits are
listed in \cref{table:netstats}. Our model explains the emergence of heavy
tailed indegree distributions through a \textit{local} process that
adjusts bias towards linking to well-connected nodes

\begin{table}
 \center
 {
  \begin{tabular}[c]{lrrrr} \toprule
  Network Dataset &  \texttt{LN} $(\mu, \sigma)$ & \texttt{DPL} $\alpha$       &  Avg. ${\texttt{LCC}}$  & \texttt{AA} $r$   \\ \midrule
  \texttt{USSC}     &   (1.19, 1.18) & 2.32     & 0.12    & -     \\
  \texttt{HEP-PH}   &   (1.32, 1.41) & 1.67     & 0.12    & -     \\
  \texttt{Semantic} &   (1.78, 0.96)  & 1.58     & 0.06    & -     \\   \midrule
  \texttt{ACL}      &   (1.93, 1.38)  & 1.43     & 0.07    & 0.07     \\
  \texttt{APS}      &   (1.62, 1.20)  & 1.26     & 0.11    & 0.44     \\
  \texttt{Patents}  &   (1.10, 1.01)   & 1.94     & 0.04    & 0.72    \\
  % \texttt{PYPI}         & 1.208     & 0.0524    & 0.692   & a\\
   \bottomrule
  \end{tabular}
  \vspace{1mm}
  \caption{Global network properties: lognormal (\texttt{LN}) indegree distribution mean \& standard deviation $(\mu, \sigma)$,
  densification power law (\texttt{DPL}) exponent $\alpha$, average local clustering coefficient (${\texttt{LCC}}$)
  and attribute assortativity (\texttt{AA}) coefficient of six network datasets.}
  \label{table:netstats}
 }
\end{table}

\textbf{Triadic Closure \& Clustering}
In the triadic closure phenomenon \cite{simmel1950sociology,
newman2001clustering}, nodes with common neighbor(s) have an increased
likelihood of forming a connection.
% Empirical studies
% \cite{kossinets2006empirical} show that the probability of edge formation
% increases with the number of common neighbors.
The local clustering coefficient
of a node measures the prevalence of triadic closure in its neighborhood; It is
the probability that two randomly chosen neighbors of the node $i$ are
connected. In directed networks, the neighborhood of a node $i$ can refer to the
set of nodes that link to $i$, set of nodes that $i$ links to or the union of
both sets. We define the neighborhood to be the set of all nodes that link to
node $i$. Real-world networks tend to exhibit high average local clustering, as
shown in \cref{table:netstats}. However, average local clustering is not a
representative statistic of the \textit{skewed} local clustering distributions
shown in \Cref{fig:cc_dc}. Furthermore, real-world networks exhibit a negative
correlation between node indegree  and local clustering. As shown in
\cref{fig:cc_dc}, the average local clustering  decreases as indegree increases.
That is, low indegree nodes have small, tightly knit neighborhoods and high
indegree nodes tend have large, star-shaped neighborhoods. We propose a model
that explains how clustering in real-world networks can arise from local processes
of exploration \& link formation.

\begin{figure}
 \centering
 \includegraphics[width=\columnwidth]{cc_analysis2}
 \caption{
    Local clustering in real-world networks have common characteristics:
    skewed local clustering distribution (left subplot) and a negatively correlated
    relationship between indegree and average local clustering (right subplot).
 }
 \label{fig:cc_dc}
\end{figure}


% Homophily and Assortativity
\textbf{Homophily \& Global Assortativity}
Real-world attributed networks tend to exhibit homophily
\cite{mcpherson2001birds}, the phenomenon in which similar nodes are more likely
to be connected than dissimilar nodes. The assortativity coefficient
~\cite{newman2002assortative} $r \in [-1, 1]$, defined as the ratio between the observed modularity and
the maximum possible modularity with respect to set of attribute values $B=\{b_1...b_l\}$,
quantifies the level of homophily in an attributed network. Intuitively, it
compares the observed fraction of edges between nodes with the same attribute
value to the expected fraction of edges between nodes with same attribute value
if the edges were rewired randomly.
Attributed networks \texttt{ACL}, \texttt{APS} \& \texttt{Patents} exhibit
varying level of homophily, as shown in \Cref{fig:mixing}, with assortativity
coefficient ranging from $0.07$ to $0.72$.
The magnitude of the attribute assortativity
signifies the extent to which attribute similarity influences edge formation.
We embed attribute based preferences at the local level lead to generate networks
with varying attribute mixing patterns. 

\begin{figure}
 \centering
 \includegraphics[width=\columnwidth]{mixing_v3}
 \caption{
    Attributed networks exhibit varying levels of homophily. The subplots
    illustrate the mixing patterns in \texttt{ACL}, \texttt{APS} and \texttt{Patents}
    w.r.t. attributes \texttt{Venue} ($r=0.07$), \texttt{Journal} ($r=0.44$) and
    \texttt{Category} ($r=0.72$) respectively.
 }
 \label{fig:mixing}
\end{figure}

\textbf{Increasing Outdegree over Time}
The average outdegree of nodes that join real-world networks tends to increase
as functions of network size and time. This phenomenon densifies networks and shrinks their
diameter over time; Leskovec et al. \cite{leskovec2005graphs} show that
densification in many real networks exhibit a power law relationship between the
number of edges $e(t)$ and nodes $n(t)$ at time $t$: $e(t) \propto
n(t)^{\alpha}$. \Cref{table:netstats} lists the densification power law exponent $\alpha$ in
the network datasets. In our proposed model, we increase the outdegree of incoming nodes
at a linear or superlinear rate to account for the accelerated network growth
observed in real networks.

To summarize, factors such as preferential attachment, triadic closure and homophily not only effect how individuals
form connections at the local level but also explain regularities in
global structural properties of real-world networks.
Next, we discuss empirical studies in sociology that examine network formation
and decision making.

\begin{figure*}[b]
    \centering
    \includegraphics[width=.95\linewidth]{rw_diag}
    \caption{Edge formation in \texttt{ARW}: consider
    an incoming node $u$ with outdegree ${m=3}$ and attribute value {$B(u)=\textsc{red} \in \{\textsc{red},\textsc{green}\}$}.
    In fig. 3a, $u$ joins the network and selects seed $v_a$ via \textsc{Select-Seed}.
    Then, in fig. 3b, $u$ initiates a \textsc{Random-Walk} and traverses from $v_a$ to $v_b$ to $v_c$.
    Finally, $u$ jumps back to its seed $v_a$ and restarts the walk, as shown in fig. 3c.
    Node $u$ halts the random walk after linking to $v_a$, $v_c$ \& $v_d$.
    }
    \label{fig:randomwalk}
\end{figure*}


\subsection{Insights from Sociological Studies}

Sociological studies on network formation seek to explain
how individuals form edges in real-world networks.
Empirical studies
\cite{35626,block2014multidimensional} that investigate
the interplay between triadic closure and homophily in evolving networks
indicate that \textit{both} structural proximity and homophily are statistically
significant factors that simultaneously influence edge formation. While homophilic preferences
\cite{mcpherson2001birds} induce edges between similar nodes,
structural factors (e.g. network distance) act as constraints that restrict
edge formation to structurally proximate nodes (e.g. friend of a friend).
Furthermore, extensive work \cite{simon1972theories,gigerenzer1996reasoning,lipman1995information}
on individual decision making have established that individuals are \textit{boundedly}
rational actors. That is, individuals make decisions under constraints of limited
information, cognitive capacity and time. This implies that individuals that join networks
employ simple rules to form edges under constraints of limited information and partial network access.
For example, a researcher cites academic papers without knowledge of or access
to the entire literature in her or his field.

Based on these observations, a faithful characterization of edge formation in real-world
networks necessitates bias towards nodes that are similar, proximate or
well-connected under constraints of limited information and network access.
Existing preferential attachment \& fitness-based models
\cite{dorogovtsev2000structure,kim2017effect,singh2017relay,barabasi1999emergence}
make two assumptions that are inconsistent with the findings of aforementioned
empirical studies.
First, by assuming that successive edge formations are independent, these model
disregard the effect of triadic closure and structural proximity. Second, they
implicitly require incoming nodes to have complete network access (e.g., connect
to any node) or explicit knowledge of one or more properties (e.g., fitness) of
every node in the network.

To summarize, citation networks tend to be homophilic networks that
undergo accelerated network growth and exhibit regularities in structural
properties: heavy tailed indegree distribution, skewed local clustering distribution,
negatively correlated degree-clustering relationship and varying attribute mixing patterns.
These global properties are a cumulative effect of resource constrained edge formation decisions.

Next, we propose a growth model that unifies multiple
sociological phenomena to explain how \textit{local} factors that affect
edge formation lead to the emergence of global structural properties
observed in real-world networks.

