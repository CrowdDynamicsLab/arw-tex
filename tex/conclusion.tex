
%!TEX root = draft.te
\section{Conclusion}
\label{sec:Conclusion}
In this paper, we develop a network growth model that explains the
structure of attributed networks through a local edge formation mechanism. Our
model \texttt{ARW} is normative, accurate and simple. We incorporated multiple
sociological phenomena into our model to intuitively prototype how individuals
form edges under constraints of limited information and partial network access.
Through our experiments, we validated the efficacy of our model in jointly preserving
multiple structural properties of real-world networks. We also showed that our
model can preserve local assortativity distributions of attributed networks.
% Furthermore, we discussed the weaknesses of global processes such as
% preferential attachment \& triangle closing and addressed the limitations of our
% model.

Our work signifies the need to understand how local processes of link formation
give rise to structural characteristics of real-world networks.
We identify three future directions: understanding the emergence of higher-order clustering [49]
through local processes, modeling the effect of homophily on the formation of temporal motifs [38]
and learning individual preferences in evolving networks with multiple attributes.

% In this paper, we model resource-constrained network growth model in which nodes
% use a random walk process to form edges under constraints of limited information
% and network access constraints. The problem is important because edge formation
% in real networks is usually a local process. In typical network growth
% scenarios, nodes in the network either have limited information about the other
% nodes in the network or the system allows access to only restricted portion of
% the existing network. It therefore becomes imperative to model how the local
% processes of link formation gives rise to network characteristics. In this work,
% we show that multiple structural properties of real networks can arise from the
% local process of exploration and link formation. Our results indicate significant
% improvement over the next best competing model \textsc{HZ}
% \cite{herrera2011generating} by a significant margin.