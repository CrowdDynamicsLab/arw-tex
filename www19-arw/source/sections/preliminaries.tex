%!TEX root = draft.tex

\section{Preliminaries}
\label{sec:Preliminaries}

In this section, we describe three edge formation processes that influence
individuals' decisions in real networks and outline the network datasets used in this paper.


\subsection{Edge Formation Processes} \label{sub:Structural Properties}
We describe three edge formation processes --- preferential attachment,
triadic closure \& homophily --- that influence how individuals form links in real
networks. Compact statistical descriptors of global network properties ~\cite{newman2010networks}
--- degree distribution, local clustering coefficient \& assortativity --- quantify the cumulative effect
of these processes on global network structure.


The preferential attachment phenomenon \cite{simon1955class,
barabasi1999emergence} states that nodes with higher degree receive links at a
faster rate because incoming nodes tend to link to well-connected nodes that
have more visibility. As a result, initial
differences in node connectivity get reinforced over time, giving rise to a
rich-get-richer effect. This phenomenon cumulatively leads to a heavy tailed
degree distribution, in which a small but significant fraction of nodes turn
into well-connected hubs. Empirical measurements on forty-seven networks
\cite{kunegis2013preferential} indicate that presence of \textit{sublinear}
preferential attachment in most cases.

% clustering

Triadic closure \cite{simmel1950sociology, newman2001clustering} is the phenomenon in which two nodes with a
common neighbor have an increased likelihood of forming a connection.
Empirical studies \cite{kossinets2006empirical} show that the probability of edge formation
increases with the number of common neighbors in information, social and citation networks. The local clustering coefficient
$C_i$ of node $v_i$ measures the prevalence of triadic closure in the
neighborhood $N_i$; It is the probability that two randomly chosen neighbors, $v_j$ and $v_k$,
of node $v_i$ are connected:
$$ C_i = \frac{|\{e_{jk} : v_j \in N_i, v_k \in N_i, e_{jk} \in E\}|}{|N_i|(|N_i|-1)}$$
In directed networks, the neighborhood of node $v_i$ can refer to the set of
nodes that link to $v_i$, set of nodes that $v_i$ links to or the union of
both. We define the neighborhood $N_i$ to be a set of all nodes that link to
node $v_i$.

% Homophily and Assortativity
Real attributed networks tend to exhibit homophily
\cite{mcpherson2001birds}, the phenomenon in which similar nodes are more likely
to be connected than dissimilar nodes. Homophilic preferences at the local level
can lead to modular networks that have relatively dense clusters of nodes that are similar
to each other. The assortativity coefficient $r$~\cite{newman2002assortative},
which measures the level of homophily (or heterophily) in a directed network with
categorical nodal attribute $A$, is the ratio between the
observed modularity $Q_A$ and the maximum possible modularity $Q^{\max}_A$:
$$ r = \frac{Q_A}{Q^{\max}_A} = \frac{\sum\limits_{i \in A} e_{ii} - \sum\limits_{i \in A} e_{i.}e_{.i}}{1 - \sum\limits_{i \in A} e_{i.}e_{.i}} $$
Modularity $Q_A$ compares the observed fraction of edges between nodes with the same attribute
value $\sum_i e_{ii}$ to the expected fraction of edges between nodes with same
attribute value if the edges were rewired randomly:  $\sum_i e_{i.}e_{.i}$.
High assortativity implies that nodes with the same
attribute value are more likely to be connected than nodes with different attribute values.

% conclude
We use global network properties to identify common structural
characteristics of real networks and incorporate these processes into our model
in~\Cref{sec:Proposed Model} and~\Cref{sec:Empirical Analysis} respectively.
